\documentclass[openany]{article}
\usepackage{xcolor}
\usepackage{makeidx}
\usepackage{geometry}       
\usepackage[chorded]{songs} 
%\usepackage[parfill]{parskip}  
\usepackage{graphicx}
%\usepackage{lipsum}
%\usepackage{amssymb}
%\usepackage{epstopdf}
%\DeclareGraphicsRule{.tif}{png}{.png}{`convert #1 `dirname #1`/`basename #1 .tif`.png}
%\geometry{letterpaper}    

\usepackage[colorlinks=true, pdfstartview=FitV, linkcolor=blue, 
            citecolor=blue, urlcolor=blue]{hyperref}
            
\setlength{\topmargin}{-0.5cm}
\setlength{\headheight}{0cm}
\setlength{\headsep}{-0.5cm}
\setlength{\textheight}{25cm}
\setlength{\textwidth}{18cm}
\setlength{\oddsidemargin}{0cm}
\setlength{\evensidemargin}{0cm}
\setlength{\parindent}{0.25cm}
%\setlength{\parskip}{0.25cm}



%\includeonly{Chapter1}


%\newtheorem{theorem}{Theorem}
%\newtheorem{corollary}[theorem]{Corollary}
%\newtheorem{definition}{Definition}
%\newtheorem{lemma}{Lemma}
%\newtheorem{exercise}{Exercise}
%\newtheorem{remark}{Remark}
%\newtheorem{example}{Example}
%\newtheorem{warning}{Warning}


\renewcommand\printchord[1]{{\color{red!70!black}#1}}




% ------------------- Title and Author -----------------------------
\title{$Bokt^{h}or$ leent weleens wat!}
\author{Tom, Martin, Coen}
\newindex{titleidx}{titleidx}
\newauthorindex{authidx}{authidx}

\begin{document}


%\frontmatter
\maketitle
%\includegraphics[width=\textwidth]{pics/boktor_logo2}

\tableofcontents
\showindex{Borrowed}{authidx}
%\showindex{Al hun liedjes}{titleidx}

%\showindex[2]{Nummers}{titleidx}
%\showindex[2]{artiesten}{authidx}


	% eigen nummers
%\songsection{Eigen nummers}
\begin{songs}{} %{titleidx,authidx}


\songcolumns{1}
%\begin{Large}

\renewcommand{\lyricfont}{\sffamily\large}

%\spenalty=-10000
%\afterpreludeskip=2pt plus 1fil
%\beforepostludeskip=2pt plus 1fil



	% en dan het gejatte werk
\setcounter{songnum}{666}

	% Activate the following line by filling in the right side. If for example the name of the root file is Main.tex, write
% "...root = Main.tex" if the chapter file is in the same directory, and "...root = ../Main.tex" if the chapter is in a subdirectory.
 
%!TEX root =  

\beginsong{Basket Case (in E)}[by={Green Day}]

\beginverse
\[E]Do you have the \[B]time to \[C#m]listen to me \[Abm]whine	\\
A\[A]bout nothing and \[E]everything all at \[B]once?	\\
I am one of those melodramatic fools	\\
Neurotic to the bone, no doubt about it	
\endverse


\beginchorus
Sometimes I give myself the creeps	\\
Sometimes my mind plays tricks on me	\\
It all keeps adding up, I think I'm cracking up	\\
Am I just paranoid? Am I just stoned?	
\endchorus


\beginverse
I went to a shrink to analyze my dreams	\\
She says it's lack of sex that's bringing me down	\\
I went to a whore, she said my life's a bore	\\
So quit my whining 'cause it's bringing her down	
\endverse


\beginchorus
Sometimes I give ...
\endchorus


\beginverse
Grasping to control	\\
So I better hold on	
\endverse


\beginchorus
Sometimes I give ...
\endchorus


\endsong
		
	% Activate the following line by filling in the right side. If for example the name of the root file is Main.tex, write
% "...root = Main.tex" if the chapter file is in the same directory, and "...root = ../Main.tex" if the chapter is in a subdirectory.
 
%!TEX root =  

\beginsong{Learn to fly }[by={Foo Fighters}]

\gtab{B2}{224422}
\gtab{F#m4/7*}{044200}


\beginverse
\[B F#m E]
\[B2]Run and tell all of the \[F#m4/7*]angels, \brk \[E]This could take all night \\
\[B]Think I need a devil to \[F#m]help me get things \[E]right \\
\[B]Hook me up a new revo\[F#m]lution, \brk Cause \[E]this one is a lie \\
We \[B]sat around laughing and \[F#m]watched the last one \[E]die 
\endverse


\beginchorus
I'm looking to the sky to save me, \brk Looking for a sign of life \\
Looking for something to help me burn out bright \\
I'm looking for a complication, \brk Looking cause I'm tired of trying \\
\[G]Make my way back \[A]home when I learn to fly \[B]high 
\endchorus


\beginverse
I think I'm done nursing the patience, \brk It can wait one night \\
I'd give it all away if you give me one last try \\
We'll live happily ever trapped, \brk If you just save my life \\
Run and tell the angels that everything's alright 
\endverse


\beginchorus
\emph{chorus}
\emph{repeat last line:} Make my way back home when I learn to. . .  
\endchorus


\beginverse
Fly along with me, I can't quite make it alone \\
Try to live this life my own (and) \\
Fly along with me, I can't quite make it alone \\
Try to live this life my own. . .  
\endverse

%\repchoruses

\beginchorus
\emph{chorus (2x)}, then end with: \[E A B]
\endchorus





\endsong
		
	% Activate the following line by filling in the right side. If for example the name of the root file is Main.tex, write
% "...root = Main.tex" if the chapter file is in the same directory, and "...root = ../Main.tex" if the chapter is in a subdirectory.
 
%!TEX root =  

\songcolumns{2}

\beginsong{Big me (Foo Fighters)}


\gtab{CaddG}{X32013}
\gtab{Am7}{X02213}
\gtab{G}{320033}


\beginverse
When I \[Cadd9]talk about it,	\\
It \[Am7]carries on,	\\
\[G]Reasons only \[F]knew.	\\
When I \[Cadd9]talk about it,	\\
\[Am7]Aries or \[G]treasons all \[F]renew.	
\endverse


\beginchorus
\[E]Big me to \[F]talk about it.	\\
\[C]I could stand to \[C7]prove.	\\
\[E] If we can \[F]get around it,	\\
\[C]I know that it's \[G]true.	
Well I \[C]talked about it,	\\
\[Am7]Carried on,	\\
\[G]Reasons only \[F]knew,	\\
But it's \[Cadd9]you I \[G]fell in\[F]to.
\endchorus


\beginverse
Well I talked about it,	\\
It carries on,	\\
Reasons only knew.	\\
When I talk about it,	\\
Aries or treasons all renew.
\endverse


\beginchorus
Big me to talk about it.	\\
I could stand to prove.	\\
If we can get around it,	\\
I know that it's true.
Well I talked about it,	\\
Put it on.	\\
Never was it true,	
But it's you I fell into.
\endchorus

\beginverse
Well I talked about it,	\\
Put it on.	\\
Never was it true,	
But it's you I fell into.	\\
I fell into.	\\
I fell into.	
\endverse


\endsong

\songcolumns{1}

		
	% Activate the following line by filling in the right side. If for example the name of the root file is Main.tex, write
% "...root = Main.tex" if the chapter file is in the same directory, and "...root = ../Main.tex" if the chapter is in a subdirectory.
 
%!TEX root =  

\beginsong{I Alone (LIVE)}

\beginverse
it's easier not to be wise	\\
and measure these things by your brains	\\
I sank into Eden with you	\\
alone in the church by and by	\\
I'll read to you here, save your eyes	\\
you'll need them, your boat is at sea	\\
your anchor is up, you've been swept away	\\
and the greatest of teachers won't hesitate	\\
to leave you there, by yourself, chained to fate
\endverse


\beginverse
I alone love you,	\brk I alone tempt you	\\
I alone love you,	\brk fear is not the end of this!	
\endverse


\beginverse
it's easier not to be great	\\
and measure these things by your eyes	\\
we long to be here by his resolve	\\
alone in the church by and by	\\
to cradle the baby in space	\\
and leave you there by yourself chained to fate	
\endverse


\beginverse
oh, now, we took it back too far,	\\
only love can save us now,	\\
all these riddles that you burn	\\
all come runnin' back to you,	\\
all these rhythms that you hide	\\
only love can save us now,	\\
all these riddles that you burn yeah, yeah, yeah	
\endverse




\endsong
		


		\beginscripture{()}
			Wat betreft $1001$ ev.: er was eens ...
			(talking about vagevuur) 
		\endscripture



% keep on dreaming, misschien als we een keer een echte zanger(es) hebben!? 
\setcounter{songnum}{1001}  
	% Activate the following line by filling in the right side. If for example the name of the root file is Main.tex, write
% "...root = Main.tex" if the chapter file is in the same directory, and "...root = ../Main.tex" if the chapter is in a subdirectory.
 
%!TEX root =  


\beginsong{All over you }[by={(LIVE)}]

\gtab{G}{320033}
\gtab{G2}{355433}
%\gtab{D}{XX0232}
\gtab{F}{133200}
%\gtab{F#}{244322}
\gtab{A}{X02220}
%\gtab{G#}{466544}
\gtab{Bm}{X24432}
\gtab{C#m}{X46654}
\gtab{A2}{577655}
\gtab{E}{X7999X}
\gtab{Dsus2}{X57755}
\gtab{Csus2}{X35533}


\beginverse*
intro \rep{2}:   \[ G2 \ F# \ Bm \ A2 \ Dsus2  ]
\endverse


\beginverse
\[D] Our love is \[A]like water,  \brk \[F#] Pinned down and ab\[G]used for \[A]being \[D]strange	
\[D] Our love is \[A] no other \[F#] then me a\[G]lone, for \[A]me all \[D]day	
Our love is \[A]like water, \[F#] pinned down and a\[G]bused, hey \[A]hey
\endverse

\beginchorus
\[G2] All \[F#]over \[Bm]you, all \[A2]over \[Dsus2]me	\brk The sun, the fields, the sky	
\[G2]I've \[F#]often \[Bm]tried to \[A2]hold  \brk The \[Dsus2]sea, the sun, the fields, the tide	
\[G2]Pay \[F#]me \[Bm]now, \[G2]Pay \[F#]me \[Bm]now, \brk Oh, \[E]yeah  \[A2]
\endchorus

\chordsoff
\beginverse
\[D] Our love is \[A]like water,  \brk \[F#] Pinned down and ab\[G]used for \[A]being \[D]strange	
\[D] Our love is \[A] no other \[F#] then me a\[G]lone, for \[A]me all \[D]day	
Our love is \[A]like angels, \[F#] pinned down and a\[G]bused, hey \[A]hey
\endverse

\beginchorus
\[G2] All \[F#]over \[Bm]you, all \[A2]over \[Dsus2]me	\brk The sun, the fields, the sky	
\[G2]I've \[F#]often \[Bm]tried to \[A2]hold  \brk The \[Dsus2]sea, the sun, the fields, the tide	
\[G2]Pay \[F#]me \[Bm]now, \[G2]lay \[F#]me \[Bm]down, , \brk  \[G2] Pay me \[F#]now, lay me \[Bm]down, 
\[Csus2]Lay me down, lay me down, lay me down	
\[G2] All \[F#]over \[Bm]you, all \[A2]over \[Dsus2]me	, \brk  \[G2] All \[F#]over \[Bm]you, all \[A2]over \[Dsus2]me, yeah
\[G2]Pay \[F#]me \[Bm]now, \[G2]lay \[F#]me \[Bm]down, down, , \brk  \[G2] Pay me \[F#]now, lay me \[Bm]down, 
\[Csus2]Lay me down, lay me down, laaaay.	
\endchorus

\beginverse
\begin{small}
\begin{verbatim}
bridge (6x):
E--------------------0------------------------------
B----------------------0----------------------------
G-----------------2------2--------------------------
D--------------------------3------------------------
A---------------------------------------------------
E--0---0-2-0-2-1------------------------------------
\end{verbatim}
\end{small}
\endverse


\beginverse
\[D] Our love is \[A]like water,  \brk \[F#] Pinned down and ab\[G]used for \[A]being \[D]strange	
\[D] Our love is \[A] no other \[F#] then me a\[G]lone, \brk hey, hey, hey	
\endverse

\beginchorus
\[G2] All \[F#]over \[Bm]you, all \[A2]over \[Dsus2]me	\brk The sun, the fields, the sky	
\[G2]I've \[F#]often \[Bm]tried to \[A2]hold  \brk The \[Dsus2]sea, the sun, the fields, the tide	
\[G2]Pay \[F#]me \[Bm]now, \[G2]lay \[F#]me \[Bm]down, , \brk  \[G2] Pay me \[F#]now, Pay me \[Bm]now, , \brk  \[Csus2]Lay me down, lay me down, laaaay.	
\endchorus

\chordson

\beginchorus
OUTTRO: \brk  \[A2  G\#  C\#m ]  Hey hey hey,
\[A2  G\#  C\#m ]  yeah yeah yeah yeah yeah yeah yeah 
\[A2  G\#  C\#m ]  Hey hey yaaah ooooh, \brk  \[G2 F# Bm A2 Dsus2 ]
\endchorus


\endsong
		
	% Activate the following line by filling in the right side. If for example the name of the root file is Main.tex, write
% "...root = Main.tex" if the chapter file is in the same directory, and "...root = ../Main.tex" if the chapter is in a subdirectory.
 
%!TEX root = ../boktor.tex 

\beginsong{Hold On}[by=Tom Waits]


%{\nolyrics Intro: \[E] \[G] \[A]}

\beginverse
They hung a sign up in our town, \brk "if you live it up, you won't \brk  live it down"
So, she left Monte Rio, son, \brk  Just like a bullet leaves a gun
With charcoal eyes and Monroe hips, \brk  She went and took that California trip
Well, the moon was gold, her, \brk  Hair like wind
She said don't look back just, \brk  Come on Jim
Oh you got to, \brk  Hold on, Hold on, \brk  You got to hold on
Take my hand, I'm standing right here, \brk  You gotta hold on
\endverse

\beginverse
Well, he gave her a dimestore watch, \brk  And a ring made from a spoon
Everyone is looking for someone to blame, \brk  But you share my bed, you share my name
Well, go ahead and call the cops, \brk  You don't meet nice girls in coffee shops
She said baby, I still love you, \brk  Sometimes there's nothin left to do
Hold on ...
\endverse

\beginverse
Well, God bless your crooked little heart St. Louis got the best of me
I miss your broken-china voice, \brk  How I wish you were still here with me
Well, you build it up, you wreck it down, \brk  You burn your mansion to the ground
When there's nothing left to keep you here, when, \brk  You're falling behind in this, \brk  Big blue world
Hold on ...
\endverse

\beginverse
Down by the Riverside motel,, \brk  It's 10 below and falling
By a 99 cent store she closed her eyes, \brk  And started swaying
But it's so hard to dance that way, \brk  When it's cold and there's no music
Well your old hometown is so far away, \brk  But, inside your head there's a record, \brk  That's playing, a song called
Hold on ...
\endverse


\endsong
		
	% Activate the following line by filling in the right side. If for example the name of the root file is Main.tex, write
% "...root = Main.tex" if the chapter file is in the same directory, and "...root = ../Main.tex" if the chapter is in a subdirectory.
 
%!TEX root = ../boktor.tex 

\beginsong{You don't know }[by=Milow]


%{\nolyrics Intro: \[E] \[G] \[A]}



%\beginverse
%\endverse


\beginverse
G = 320033
C = 032010
E = 022100
E7 = 020100

*** CAPO IV ***

Intro:  Am F G Am F G C E


Am         F                            G
sometimes everything seems awkward and large
Am                              F
imagine a Wednesday evening in March
           G           C    E
future and past at the same time
Am                F                            G
I make use of the night, start drinking a lot
             Am                                F   G
although not ideal for now it's all that I've got
              C         E
it's nice to know your name

Am             F                G
you don't know you don't know
               Am             F   E
you don't know anything about me


Am         F                      G
an ocean a lake I need a place to drown
                 Am                         F     G
let's freeze the moment because we're going down
         C          E
tomorrow you'll be gone
Am                  F                   G
you're laughing too hard this all seems surreal
       Am                       F
I feel peculiar now what do you feel
                       G                  E7     E
do you think there's a chance that we can fall

Am             F               G
you don't know you don't know
               Am              F
you don't know anything about me
        G                  C      E
what do I know I know your name
Am             F               G
you don't know you don't know
Am                            F   G
you don't know anything about me
E7      E
anymore

F          G              C
I gave up dreaming for a while
F          G              C
I gave up dreaming for a while


Am                 F               G
I've noticed these are mysterious days
             Am                        F
I look at it like a jigsaw puzzle and gaze
               G                 C     E
with wide open mouth and burning eyes
Am       F               G
if only I could start to care
Am                                      F      G
my dreams and my Wednesdays ain't going nowhere
E7
baby baby baby you don't know


Am             F                G
you don't know you don't know
               Am             F
you don't know anything about me
        G                  C      E
what do I know I know your name
Am             F               G
you don't know you don't know
               Am G           F    F
you don't know anything about me


Am F G Am F G C E
Am F G Am F G E7 E
\endverse

\endsong

	
	
%\end{Large}
\end{songs}


\bibliographystyle{plain-annote}
\bibliography{mybibliography}

\end{document}
\end

