% Activate the following line by filling in the right side. If for example the name of the root file is Main.tex, write
% "...root = Main.tex" if the chapter file is in the same directory, and "...root = ../Main.tex" if the chapter is in a subdirectory.
 
%!TEX root = ../boktor.tex 

\beginsong{Fouragemeesterstent}[by=]


%{\nolyrics Intro: \[E] \[G] \[A]}


\beginverse
Er is nog soep, soep, soep,  \brk  voor de hongerige troep.
in de tent, in de tent, \brk  in de tent, in de tent.
Er is nog soep, soep, soep,  \brk  voor de hongerige troep.
In de fouragemeesters tent.
\endverse

\beginchorus
Dat wist ik niet en bovendien,
dat kan ik zonder bril niet zien.
Dat kan ik zon-der bril niet zien.
\endchorus

\beginverse
Er is nog pap, pap, \brk  voor iedereen een hap.
In de tent, in de tent, \brk  In de tent, in de tent.
Er is nog pap, pap, \brk  voor iedereen een hap.
In de fouragemeesters tent.
\endverse

\beginchorus
Er is nog kaas, kaas, \brk  zo oud als Sinterklaas.
in de tent, in de tent, \brk  in de tent, in de tent.
Er is nog kaas, kaas, \brk  zo oud als Sinterklaas.
In de fouragemeesters tent.
\endchorus

\beginverse
\musicnote{En vervolgens ga je verder met:}
Brood - voor iedereen een moot. - gevonden in de sloot.
Koek - voor een hongerige troep.
Drop - voor een verkouden hop.
Ham - voor op de boterham.
Vla - van de sokken van m'n ma.
Thee - al uit onze w.c.
Melk - en dat is goed voor elk.
Bier - van een pas gemolken stier. - voor een helehoop plezier.
\endverse






%\beginchorus
%\endchorus


%\beginverse
%\endverse

\endsong
