% Activate the following line by filling in the right side. If for example the name of the root file is Main.tex, write
% "...root = Main.tex" if the chapter file is in the same directory, and "...root = ../Main.tex" if the chapter is in a subdirectory.


\beginsong{Gaode Mee, Hee }[by=Gerard van Maasakkers]
%\gtab{E7sus4}{0320202}
%\gtab{E7}{031020}
%\gtab{D/F#}{200232}
%\gtab{Dm/B}{X20231}
%\gtab{Dm6/A}{X00201}
%\gtab{E7/Ab}{123121}
%\gtab{E7/Ab}{XXX130}
%\gtab{Ami}{5XX555}
%\gtab{Esus4/Ab}{XX3122}

%\transpose{-5}
\begin{Large}

\beginchorus
intro: Refrein zonder tekst.
\endchorus

\beginchorus
\[G]Hee, gaode \[D]mee, dan \[Em]gaon we'n eindje \[Bm]lopen.
\[C]Hou toch op mee poetsen, \[D]kijk toch nie zo \[G]nauw  \[D] .
\[G]Hee gaode \[D]mee, de \[Em]bluumkes staon weer \[Bm]open.
\[C]Laot oewe jas mer \[D]hangen, 'tis nie \[G]kou \[G] .
\endchorus

\beginverse
\[Em] Ik weet 'n plaatske in 't \[Bm]Nuenens Broek.
\[Em] D'r is nog \[D]niemes ooit gewist.
\[Em] Wij gaon d'r soamen nou 's \[Bm]op bezoek.
\[Em] Dan maken \[G]wij mee alle \[B7]plantjes,
En alle \[C]veugelkes in 't \[D]broek 'n hul groot \[G]fist.
\endverse

\beginchorus
\[G]Hee, gaode \[D]mee, ...
\endchorus

\beginchorus
tussenspel: Refrein zonder tekst.
\endchorus

\beginverse
\[Em]De zon gao nerges zo schoon \[Bm]onder as daor.
\[Em] En nerges \[D]is de lucht zo \[G]blauw.  \echo{blauw blauw \[Em]blauw}
D'r liggen nerges zoveul \[Bm]blaaikes as daor.
\[Em] Dus haalt oew \[G]schoen mer van de \[B7]zulder.
En kamt mer \[C]vlug wa dur oew \[D]haor, dan gaon we \[G]gauw.
\endverse

\beginchorus
\[G]Hee, gaode \[D]mee, ...
\endchorus

\beginchorus
Slot: Laatste regel herhalen.
\endchorus

\end{Large}

%\chordsoff

\chordson
\endsong
