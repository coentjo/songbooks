% Activate the following line by filling in the right side. If for example the name of the root file is Main.tex, write
% "...root = Main.tex" if the chapter file is in the same directory, and "...root = ../Main.tex" if the chapter is in a subdirectory.
 
%!TEX root = ../boktor.tex 

\beginsong{Het land van Maas en Waal }[by=vergeetmenietje,cr={\copyright~2015 Music Inc.}]


%{\nolyrics Intro: \[E] \[G] \[A]}


\beginchorus
Refrein:
Onder de groene hemel, in de blauwe zon
Speelt het blikken harmonieorkest in een grote regenton
Daar trekt over de heuvels en door het grote bos
De lange stoet de bergen in van het circus Jeroen Bosch
En we praten en we zingen en we lachen allemaal,
Want daar achter de hoge bergen ligt het land …
van Maas en Waal
\endchorus


\beginverse
Ik loop gearmd met een kater voorop,
Daarachter twee konijnen met een trechter op hun kop
En dan de grote snoeshaan, die legt een glazen ei
Wanneer je 't schudt dan sneeuwt het op de Egmondse abdij
\endverse

\beginverse
Ik reik een meisje mijn koperen hand,
Dan komen er twee moren met hun zwepen in de hand.
Dan blaast er een fanfare ter ere van de schaar,
Die trouwt met de vingerhoed, zij houden van elkaar
\endverse

\beginverse
Wij zijn aan de Koning van Spanje ontsnapt
Die had ons in zijn bed en de provisiekast betrapt
We staken alle kerken met brandewijn in brand
't Is koudvuur, dus het geeft niet, en het komt niet in de krant
\endverse

\beginchorus
En onder de purperen hemel, in de bruine zon
Speelt nog steeds het harmonieorkest in een grote regenton
Daar trekt over de heuvels en door het grote bos
De lange stoet de bergen in van het circus Jeroen Bosch
En we praten en we zingen en we lachen allemaal
Want achter de hoge bergen ligt het land … van Maas en Waal 
\endchorus

%\beginchorus
%\endchorus


%\beginverse
%\endverse

\endsong
