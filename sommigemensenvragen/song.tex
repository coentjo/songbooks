% Activate the following line by filling in the right side. If for example the name of the root file is Main.tex, write
% "...root = Main.tex" if the chapter file is in the same directory, and "...root = ../Main.tex" if the chapter is in a subdirectory.
 
%!TEX root = ../boktor.tex 

\beginsong{Sommige mensen vragen}[by=vergeetmenietje,cr={\copyright~2015 Music Inc.}]


%{\nolyrics Intro: \[E] \[G] \[A]}


\beginverse
Voorzingen: Sommige mensen vragen
Nazingen: Sommige mensen vragen
Voorzingen: wie wij zijn
Nazingen: wie wij zijn
Voorzingen: en waar vandaan
Nazingen: en waar vandaan
Voorzingen: dan zeggen wij:
Nazingen: dan zeggen wij:
Voorzingen: We zijn de (...) (naam van de groep invullen)
Nazingen: We zijn de (...)
Voorzingen: We komen uit (...) (plaatsnaam invullen)
Nazingen: We komen uit (...)
Voorzingen: Het knettergekke (...) (nog een keer plaatsnaam invullen)
Nazingen: Het knettergekke (...)
Voorzingen: En als z'ons niet verstaan
Nazingen: En als z'ons niet verstaan
Voorzingen: dan zingen we 't wat harder
Nazingen: dan zingen we 't wat harder


Variaties:

    het liedje een aantal keer herhalen, de eerste keer heel zachtjes, en na ieder couplet steeds harder zingen. Bij de laatste keer kun je in plaats van "dan zingen we 't wat harder" de regel "dan zijn ze doof!" zingen.
    het liedje een aantal keer herhalen, en in de laatste regel aangeven hoe het volgende couplet gezongen moet worden, dus harder, zachter, trager, sneller, hoger, lager enzovoort.
\endverse

%\beginchorus
%\endchorus


%\beginverse
%\endverse

\endsong
